\documentclass[12pt]{article}

% Encoding.
\usepackage[utf8]{inputenc}
\usepackage[T1]{fontenc}
\usepackage[english]{babel}

% Colours.
\usepackage[usenames]{xcolor}

% Listings.
\usepackage{listings}

\lstset
{
    % Language.
    language=C++,
    % Basic style.
    basicstyle=\footnotesize,
    backgroundcolor=\color{white},
    rulecolor=\color{black},
    %numbers=left,
    %numberstyle=\color{gray},
    % Other styles.
    keywordstyle=\bf\color{blue},
    commentstyle=\color{darkgray},
    stringstyle=\color{red},
    escapeinside={\%*}{*)},
    morekeywords={},
    % Misc.
    tabsize=4,
    extendedchars=true,
    breaklines=true,
    % Indention.
    xleftmargin=0pt
}

% New commands for code.
\definecolor{verylightgray}{rgb}{0.9,0.9,0.9}
\newcommand{\code}[1]{\colorbox{verylightgray}{\lstinline!#1!}}
\newcommand{\highlightlesscode}[1]{\colorbox{verylightgray}{#1}}
\newcommand{\longcode}[1]{\begin{lstlisting}!#1!\end{lstlisting}}

% Document's info.
\title{Convention Document}
\author{pr061012}
\date{Compilation date: \today}

% I fairly copied some rules from qutim's coding style
% (http://wiki.qutim.org/en/coding_style)

\begin{document}
	\maketitle

	\section{General}
		\begin{itemize}
			\item Keep It Simple, Stupid.
			\item Everything must be cool and crossplatform.
			\item Think before you code.
		\end{itemize}

	\section{Coding Convention}
		\subsection{General rules}
			\begin{itemize}
				\item Avoid short names, use only understandable names.
				\item Use single character names only for counters and temporaries.
			\end{itemize}

		\subsection{Variables and constants naming}
			\begin{itemize}
				\item Declare each variable on a separate line.
				\item Use underscores style for variables names (\code{int my_var} instead of \code{int myVar} or \code{int MyVar}).
				\item Use uppercase style for constant and defines names (\code{const int SOME_MAXIMUM = 5}).
				\item Object is an object, so the object's name must be a noun.
			\end{itemize}

		\subsection{Classes naming}
			\begin{itemize}
				\item Declare each class in separate pair of \code{*.cpp} and \code{*.h} files.
				\item Declare all class methods in header file. Code all methods implementations in \code{*.cpp}-file.
				\item Use abstract classes and inheritance not whenever it's possible, but when it's really needed.
				\item Use UpperCamelCase style for names (\code{MyClass} instead of \code{myClass}).
			\end{itemize}

		\subsection{Enumerations naming}
			\begin{itemize}
				\item Enums are classes to, so all rules in \S2.3 relate to enums too.
				\item We treat values of enum as constants, thus you should use uppercase style for enum's elements.
			\end{itemize}

		\subsection{Functions and methods naming}
			\begin{itemize}
				\item Use lowerCamelCase style for names (\code{doSomething()} instead of \code{DoSomething()})
				\item Don't omit variables names in functions declaration (\code{int sum(int a, int b);} instead of \code{int sum(int, int);}).
				\item Function does something, so the function's name must be a verb.
			\end{itemize}

		\subsection{Identation and new lines}
			\begin{itemize}
				\item Use 4-space identation.
				\item Use only one statement per line. (If you need to write a big chunk of code with similar statements per line, you're doing something \textbf{wrong}.)
				\item Use blank lines to group statements.
				\item Don't use multiple blank lines.
				\item Don't use new line for function arguments.
			\end{itemize}

		\subsection{Braces}
			\begin{itemize}
				\item Curly brace always must be on a new line.
				\item Round bracket must be on a new line if conditional statement covers several lines.
				\item Never omit curly braces for loop and conditional statements, if they contain only one line.
			\end{itemize}

		\subsection{Loop and conditional statements}
			\begin{itemize}
				\item Use several lines for loop and conditional statements if needed:
					\begin{lstlisting}
if
(
	some_condition == true &&
	checkThatEverythingIsOkay() == true
)
{
	// Do something!
}

while
(
	some_condition == true &&
	checkThatEverythingIsOkay() == true
)
{
	// Do something!
}
					\end{lstlisting}
				\item Use these one line statements only when you're \textbf{really} needed:
\begin{lstlisting}
if(some_condition) doSomething();
while(some_condition) doSomething();
\end{lstlisting}
			\end{itemize}

	\section{Commenting convention}
		\begin{itemize}
			\item Comment each function, class and namespace.
			\item Use Doxygen for class and functions commenting. Few Doxygen commenting examples:
\begin{lstlisting}
/**
 * @brief Another class description.
 */
class MyClass
{
	private:
		/// Variable.
		int var;

	public:
		/**
		 * @brief Cool MyClass constructor.
		 * @param some_var some parameter
		 */
		MyClass(int some_var);

		/**
		 * @brief  Returns var.
		 * @return var
		 */
		int getVar();
};
\end{lstlisting}
			\item Use three-slash comment (\code{///}) for variables commenting.
			\item Use multiline comment (\code{/** */}) for functions and classes.
			\item Add star char at the begining of each line in multiline comment:
\begin{lstlisting}
/**
 * @brief It's a long decscription. Really. You
 *        don't believe me?
 */
\end{lstlisting}
			\item Align all information by one column, like so:
\begin{lstlisting}
/**
 * @brief  Why you're reading this? You have
 *         nothing to do?
 * @param  one_param     one thing
 * @param  another_param another thing
 * @return returned value
 */
\end{lstlisting}
			\item Omit \code{@brief} in one-line comments for variables. Correct examples:
\begin{lstlisting}
/// Yet another variable.
int var;

/// @brief A long variable's description may contain
///        many different things.
int var;
\end{lstlisting}
				Incorrect example:
\begin{lstlisting}
/// @brief Yet another variable.
int var;
\end{lstlisting}
			\item You can use Markdown in comments if needed.
			\item Don't be affraid to describe an complicated algorithm in doxygen-comment.
\begin{lstlisting}
/**
 * @brief Very complicated function. *Algorithm*:
 *        1. One step.
 *        2. Another step.
 *        3. ???
 *        4. Maybe profit!
 */
\end{lstlisting}
			\item Create additional pages in Markdown markup with description about project's struct. (Or use something else for this, e.g. wiki).
		\end{itemize}

	\section{Git convention}
		\begin{itemize}
			\item Use git.
			\item Use different branches for different things.
			\item Create a new branch for each bug fix or new feature.
			\item Merge to \code{master} code, that \textit{really} works.
			\item Add module name, issue/bug number to commit text: \highlightlesscode{[Object] Add new attribute.} or \highlightlesscode{[Issue 24] Fix.}.
			\item Push commits to global repo as soon as possible: anyone must pull your changes.
		\end{itemize}
\end{document}
